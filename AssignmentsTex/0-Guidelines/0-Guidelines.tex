\documentclass[11pt]{amsart}
\usepackage{geometry}                % See geometry.pdf to learn the layout options. There are lots.
\geometry{letterpaper}                   % ... or a4paper or a5paper or ... 
%\geometry{landscape}                % Activate for for rotated page geometry
%\usepackage[parfill]{parskip}    % Activate to begin paragraphs with an empty line rather than an indent
\usepackage{graphicx}
\usepackage{amssymb}
\usepackage{epstopdf}
\usepackage[usenames,dvipsnames]{color}
\usepackage{hyperref}
\hypersetup{colorlinks=true}
\DeclareGraphicsRule{.tif}{png}{.png}{`convert #1 `dirname #1`/`basename #1 .tif`.png}
\renewcommand\familydefault{\sfdefault}
\newcommand{\todo}[1]{{\bf\textcolor{red}{TODO: #1}}}
\setlength{\topmargin}{0cm}
\setlength{\headheight}{0cm}
\setlength{\headsep}{1cm}
\setlength{\textheight}{7.7in}
\setlength{\textwidth}{6.5in}
\setlength{\oddsidemargin}{0cm}
\setlength{\evensidemargin}{0cm}
\setlength{\parindent}{0.25cm}
\setlength{\parskip}{0.1cm}

\usepackage{fancyhdr,graphicx,lastpage}% http://ctan.org/pkg/{fancyhdr,graphicx,lastpage}
\fancypagestyle{plain}{
  \fancyhf{}% Clear header/footer
  \fancyhead[L]{CSCI-GA.3033-018 - Geometric Modeling}% Right header
  \fancyhead[R]{\includegraphics[height=20pt]{nyu.pdf}}% Right header
  \fancyfoot[L]{\vspace{2pt} Daniele Panozzo\textbf{}}% Left footer
  \fancyfoot[R]{\vspace{2pt} \thepage}% Right footer
}

\pagestyle{plain}% Set page style to plain.
\begin{document}

\hspace{50pt}

\begin{center}

{\Huge \textbf{General Rules and Instructions}}\\
\end{center}

\section*{Plagiarism Note and Late Policy}
Copying code (either from other students or from external sources) is strictly
prohibited! We will be using automatic anti-plagiarism tools, and any violation
of this rule will lead to expulsion from the class. Late submissions will
generally not be accepted. In case of serious illness or emergency, please notify
Daniele and provide a relevant medical certificate.

\section*{Provided Libraries}
For each assignment, you will use the geometry processing library
\href{https://github.com/libigl/libigl/}{\texttt{libigl}}, which
includes implementations of many of the algorithms presented in class.
The \texttt{libigl} library
includes a set of tutorials, an introduction to which can be found in
\texttt{tutorial/tutorial.html}. You are advised to look over the relevant
tutorials before starting the implementation for the
assignments; you are also encouraged to examine the source code of all
the library functions that you use in your code to see how they were
implemented.

To simplify compilation, we will use \texttt{libigl} as a header-only library
(note that, if you prefer, you can compile it into a set of static libraries for
faster builds at your own risk---this can be brittle on some platforms). This
way, all you need to do to install \texttt{libigl} is to clone it somewhere and
point your \texttt{\$LIBIGL\_ROOT} there. If you don't want to deal with
environment variables, \texttt{libigl} can also be found if you clone it into
your assignment directory's parent directory. All dependencies of
\texttt{libigl} are included as
\href{https://git-scm.com/book/en/v2/Git-Tools-Submodules}{git submodules}, so
please follow the installation instructions below carefully (note the
`--recursive' clone).

No libraries apart from \texttt{libigl} are permitted unless permission is
granted in advance.

\section*{Installing CMake and \texttt{LIBIGL}}
Before we can begin, you must install \href{https://cmake.org/}{CMake}, the
system \texttt{libigl} uses for cross-platform builds. If you are using Linux or
macOS, I recommend installing it with a package manager instead of the
\href{https://cmake.org/download/}{CMake download page}.
E.g. on Debian/Ubuntu: \texttt{sudo apt-get install cmake} or with
\href{http://macports.org}{MacPorts} on macOS: \texttt{sudo port install cmake}.

Next, choose an installation location for \texttt{libigl}. As mentioned above,
if you clone it into the parent directory of your assignment repositories, each
assignment's CMake scripts will find it automatically. If you place it
elsewhere, you will need to set the \texttt{\$LIBIGL\_ROOT} environment variable
to point to the cloned \texttt{libigl} directory.

Finally, clone \texttt{libigl} and all of its dependencies:\\
\texttt{git clone --recursive https://github.com/libigl/libigl.git}.
The \texttt{--recursive} is needed to bring in the dependencies, which are git
submodules.

\end{document}  
